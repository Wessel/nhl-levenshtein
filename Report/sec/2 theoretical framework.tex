\section{Theoretical Framework}

\subsection{Functional Programming}
\label{ssec:fp}
Functional programming is one of four programming paradigms that treats computation as mathematical
functions where all data is immmutable.\cite{hudak1989conception} This makes functional programming
declarative. The program specifies what it should do, rather than how it should do it.

Functional programming is based on the concept of pure functions, which are functions
that have no side effects and always return the same output for the same input.

\subsubsection*{Haskell}
Haskell (or HS) is a functional programming language that is based on the lambda calculus.\cite{hutton2016programming}
Haskell is a pure functional language, which means that all functions in Haskell are pure functions.
It is also a lazy language, meaning that expressions are only ran when their values are needed.

\subsection{Imperative Programming}

Imperative programming is one of the four programming paradigms that uses statements to change
the state of a program. In imperative programming, the program is a sequence of statements that
are executed in order.\cite{syme2007introducing}

The state of the program is determined by mutable variables. Imperative programming is the
complete opposite of functional programming mentioned in \autoref{ssec:fp},
promoting the use of mutable variables.

The mutable nature of imperaetive programming makes it difficult to reason the behavior of
a function, as the state of the program can change at any time.

\subsubsection*{CSharp}
C\# (or CSharp) is an object-oriented programming language that supports imperative programming.
C\# is widely used for developing Windows applications, web applications, and games.

\subsection{The Levenshtein Distance Algorithm}
\label{ssec:t-levenshtein}

The Levenshtein Distance algorithm is an algorithm used to calculate the similarity between two
pieces of text. This is done by calculating the minimum number of single-character edits
(insertions, deletions, or substitutions) required to change one string into another.\cite{po2020similarity}

The result of the Levenshtein Distance algorithm is a value between zero and one
that represents the number of edits required to change one string into another.
This number is called the \textit{Levenshtein Ratio}.

The pseudocode for the Levenshtein Distance algorithm with a complexity of $O(m \times n)$
is shown in \autoref{alg:levenshtein}.

\begin{algorithm}
  \small
  \caption{Levenshtein Distance Algorithm\cite{po2020similarity}}
  \label{alg:levenshtein}
  \begin{algorithmic}[1]
    \Function{LevenshteinDistance}{$s, t$}
      \State $n \gets$ length of $s$.
      \State $m \gets$ length of $t$.

      \If{$n = 0$}
        \State \Return $m$
      \EndIf
      \If{$m = 0$}
        \State \Return $n$
      \EndIf

      \State Create matrix $d$ of size $0..m \times 0..n$.

      \State Initialize first row to $0..n$.
      \State Initialize first column to $0..m$.

      \State Examine each character of $s$ (i from $1$ to $n$).

      \State Examine each character of $t$ (j from $1$ to $m$).

      \If{$s[i] = t[j]$}
        \State $c \gets 0$
      \Else
        \State $c \gets 1$
      \EndIf

      \State $d[i,j] \gets \min(
        \begin{cases}
          \text{Cell above} = [i,j-1] + 1 \\
          \text{Cell left} = [di,j-1] + 1 \\
          \text{Cell diagonal} = d[i-1,j-1] + c
        \end{cases}
      )$

      \State \Return $d[n,m]$.

    \EndFunction
  \end{algorithmic}
\end{algorithm}
