\section{Conclusion}

C\# has had the best execution speed, outperforming TypeScript by $3 \times$ and Haskell by over $1000 \times$ for long-string comparisons,
showcasing that imperative programming is highly efficient for intensive tasks.

C\# did consume considerably more memory (61 MiB) and CPU usage (38.14\%).

Haskell, whilst being the slowest in execution time by far,
did use $2 \times$ less memory (39 MiB) and  $5 \times$ less CPU usage than C\# and TypeScript.
Making it an excellent choice for low resource environments where speed is not as important.

Its immutable nature and functional syntax also earned it the highest readability score (4.5),
making it an excellent choice for maintainability.

TypeScript did not excel at anything compared to C\# or Haskell,
but also did not perform poorly. Making it a good middleground for projects
on the web due to its ease of use.

These findings show that the choice of language depends on the project's priorities:
\begin{itemize}
  \item[-] \makebox[1.5cm]{\textbf{C\#}\hfill} for performance-sensitive systems
  \item[-] \makebox[1.5cm]{\textbf{TypeScript}\hfill} for small, balanced utilities
  \item[-] \makebox[1.5cm]{\textbf{Haskell}\hfill} for resource efficiency and code maintainability
\end{itemize}
